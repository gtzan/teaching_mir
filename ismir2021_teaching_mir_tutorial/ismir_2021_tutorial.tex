\documentclass[12pt]{beamer}
\usepackage{amsmath}
\usepackage{xcolor}
\usepackage{subfigure}
\usepackage{bbm} 
\usepackage{pgfpages}
\usepackage{tikz}
\usepackage{dcolumn}
\usepackage{booktabs}
\newcolumntype{M}[1]{D{.}{.}{1.#1}}

\usetikzlibrary{positioning,shapes,arrows}

\usepackage{pgf,pgfarrows,pgfnodes,pgfautomata,pgfheaps,pgfshade}
\usetheme{Air}

%\pgfpagesuselayout{4 on 1}[letter,border shrink=5mm]
%\pgfpageslogicalpageoptions{1}{border code=\pgfusepath{stroke}}
%\pgfpageslogicalpageoptions{2}{border code=\pgfusepath{stroke}}
%\pgfpageslogicalpageoptions{3}{border code=\pgfusepath{stroke}}
%\pgfpageslogicalpageoptions{4}{border code=\pgfusepath{stroke}}


\DeclareMathOperator*{\argmax}{arg\,max}

\title[ISMIR 2021 tutorial]{ISMIR 2021 Tutorial}


\logo{\pgfputat{\pgfxy(-0.5,7.5)}{\pgfbox[center,base]{\includegraphics[width=1.0cm]{figures/uvic}}}}  
\beamertemplatenavigationsymbolsempty

    \defbeamertemplate{footline}{author and page number}{%
      \usebeamercolor[fg]{page number in head/foot}%
      \usebeamerfont{page number in head/foot}%
      \hspace{1em}\insertshortauthor\hfill%
      \insertpagenumber\,/\,\insertpresentationendpage\kern1em\vskip2pt%
    }
    \setbeamertemplate{footline}[author and page number]{}



\subtitle[Teaching Music Information Retrieval]{Teaching Music Information Retrieval} 
\date[2021]{2021}
\author[G. Tzanetakis]{George Tzanetakis}
\institute[University of Victoria]{University of Victoria}
%\logo{\includegraphics[scale=.25]{unilogo.pdf}}

\begin{document}
\frame{\maketitle} % <-- generate frame with title


\AtBeginSection[]
{
\begin{frame}<beamer>[allowframebreaks]{Table of Contents}
\tableofcontents[currentsection,currentsubsection, 
    hideothersubsections, 
    sectionstyle=show/shaded,
]
\end{frame}
}



%% NOTEBOOK EXAMPLES
%% measuring_amplitude * 
%% monophonic pitch detection
%% matrix factorization
%% rhythm notation * 
%% audiolabs-erlangen.de/resources/MIR/FMP/C3/C3_MusicSynchronization.html
%% THX sound from Steve Toja MIR notebooks 
%% Lyrics classification

\section{Introduction} 

\begin{frame}{Introduction} 

Music Information Retrieval as a research field is now more than 20
years old.  I have been involved in teaching MIR in undergraduate,
graduate, and tutorial courses to students from a variety of
disciplines for over 15 years. In this tutorial we will explore
different aspects of teaching MIR and share what I have learned over
the years about how to make the teaching of MIR topics more effective.

\end{frame} 


\begin{frame}{Tutorial Structure} 

There are 6 modules each with a duration of approximately 25 minutes
followed by 5 minutes of questions. Each unit consists of a set of
slides and in some cases some associated hands-on demonstrations in
the form of Jupyter/Python notebooks. The target audience is anyone
(professor, researcher, postdoc, graduate student) who is interested
in teaching MIR. I assume that participants are familiar with the main
ideas and tasks of MIR.

\end{frame} 

\begin{frame}{Modules} 
\begin{itemize} 
\item Overview and organization
\item Adapting to a target audience and format 
\item Online learning
\item Assessment 
\item Projects 
\item Resources 
\end{itemize} 
\end{frame}


\begin{frame}{Relevant Background} 
\begin{itemize}
\item{Main focus of my research has been Music Information Retrieval (MIR)}
\item{Involved from the early days of the field} 
\item{Have published papers in almost every ISMIR conference}
\item{Organized ISMIR in 2006} 
\item{Tutorials on various MIR topics in several conferences}
\item{Taught MIR as a 4th year CS course for 12 times}
\item{Kadenze MIR program (3 courses) approximately 3000 participants} 
\end{itemize}
\end{frame}



\begin{frame}{Education and Academic Work Experience} 

\begin{itemize} 
\item{1997 BSc in Computer Science (CS), University of Crete, Greece}
\item{1999 MA in CS, Princeton University, USA} 
\item{2002 PhD in CS, Princeton University, USA} 
\item{2003 PostDoc in CS, Carnegie Mellon University, USA}
\item{2004 Assistant Professor in CS, Univ. of Victoria, Canada} 
\item{2010 Associate Professor in CS, Univ. of Victoria, Canada}
\item{2016 Professor in CS, Univ. of VIctoria, Canada}
\item{2010-2020 Canada Research Chair (Tier II) in Computer Analysis of Audio and Music} 
\item{Music theory, saxophone and piano performance, composition,
  improvisation both in conservatory and academic settings}
\end{itemize}
\end{frame} 







\begin{frame}{Work Experience beyond Academia}

  Many internships in research labs throughout studies. Several
  consulting jobs while in academia. A few representative examples:

\begin{itemize}
\item{Moodlogic Inc (2000). Designed and developed one of the earliest audio fingerprinting systems (patented) - 100000 users matching to 1.5 million songs}
\item{Teligence Inc (2005). Automatic male/female voice discrimination for voice messages used in popular phone dating sites - processing of 20000+ recordings per day.}
\item{Smule (2015-present) - various MIR related projects}
\end{itemize}
\end{frame}

\begin{frame}{Software - Marsyas} 
\begin{itemize}
\item{Music Analysis, Retrieval and Synthesis for Audio Signals} 
\item{Open source in C++ with Python Bindings} 
\item{Started by me in 1999 - core team approximately 4-5 developers}
\item{Approximately 400 downloads per month}
\item{Many projects in industry and academia} 
\item{State-of-the-art performance while frequently orders of
  magnitude faster than other systems}
\item{Not actively developed for the last 5-6 years}
\end{itemize} 
\end{frame} 

\begin{frame}{Visiting Scientist at Google Research} 

Six month study leave. Things I worked on (of course as part of larger teams):
\begin{itemize} 
\item{Cover Song Detection (applied to every uploaded YouTube video).}
\item{Audio Fingerprinting}
\item{Named inventor on 6 pending US patents related to audio matching and fingerprinting} 
\end{itemize}  
\end{frame} 




\section{Overview - Organization}


\begin{frame}{UVic MIR Course - Learning outcomes}
\begin{itemize} 
\item{Basic knowledge of DSP} 
\item{Basic knowledge of Machine Learning (ML)} 
\item{Basic knowledge of Music Theory} 
\item{Familiarity with the basic tasks that have been 
explored in MIR research and the algorithms used to solve them} 
\item{{\bf Being able to read and understand the majority 
of published literature in ISMIR}}
\item{Experience with designing and developing MIR algorithms and 
systems} 
\end{itemize} 
\end{frame}


\begin{frame}{UVic MIR Course - Lectures}
\begin{itemize} 
\item{2 lectures 1.5 hours /week}
\item{3 hours of associated homework expected for each lecture} 
\item{Total weekly commitment approximately 9-10 hours} 
\item{6 assignments each worth $10\%$ of final grade and done invididually.}
\item{Assignments typically will combine some reading and understanding of MIR literature as well as some programming of MIR algorithms} 
\item{1 final group project (2-3 students per group) $40\%$} 
\end{itemize} 
\end{frame}{}



\begin{frame}{History of MIR before computers} 
How did a listener encounter a new piece of music throughout history ? 
\begin{itemize} 
\item{Live performance} 
\item{Music Notation}
\item{Physical recording} 
\item{Radio} 
\end{itemize} 
\end{frame} 


\begin{frame}{Brief History of computer MIR}
\begin{itemize} 
\item{Pre-history ($<2000$): scattered papers in various communities. Symbolic 
processing mostly in digital libraries and information retrieval venues 
and audio processing (less explored) mostly in acoustics and DSP venues.}
\item{The birth $2000$: first International symposium on Music Information 
Retrieval (ISMIR) with funding from NSF Digital Libraries II initiative 
organized by J. Stephen Downie, Time Crawford and Don Byrd. First contact between the symbolic and the audio side.}
\item{2000-2006} Rapid growth of ISMIR 
\item{2006-2014} Slower growth and steady state
\item{2015-2021} Maturation - robust industry involvement
\end{itemize} 
\end{frame}


\begin{frame}{Teaching Organization}

  One of the main challenges of teaching MIR is how to
  structure/organize the material to be taught. A good organization
  should be comprehensive (i.e cover most MIR tasks/topics),
  incremental (concepts should be introduced gradually in a logical
  fashion), flexible (it should be possible to emphasize or skip
  different topics), and balanced.

  In the following slides I will discuss some possible ways to
  organize MIR for teaching purposes based on my extensive experience
  and experimentation.
  

\end{frame} 

\begin{frame}{Organize by stages/specificity} 
\begin{itemize}
\item{Stages}
\begin{itemize} 
\item{Representation/Hearing} 
\item{Analysis/Learning}
\item{Interaction/Action}
\end{itemize}
\item{Specificity} 
\begin{itemize} 
\item{Audio fingerprinting} 
\item{Common score performance} 
\item{Cover song detection} 
\item{Artist identification} 
\item{Genre classification} 
\item{Recommendation ? } 
\end{itemize} 
\end{itemize}
\end{frame} 

\begin{frame}{Organize based on data}
Data sources: 
\begin{itemize} 
\item{Audio}
\item{Track metadata}
\item{Score} 
\item{Lyrics} 
\item{Reviews} 
\item{Ratings} 
\item{Download patterns} 
\item{Micro-blogging}
\end{itemize} 
\end{frame} 


\begin{frame}{MIR Tasks} 
\begin{itemize} 
\item{Similarity retrieval, playlists, recommendation} 
\item{Classification and clustering} 
\item{Tag annotation}
\item{Rhythm, melody, chords}
\item{Music transcription and source separation} 
\item{Query by humming} 
\item{Symbolic MIR} 
 \item{Segmentation, structure, alignment} 
\item{Watermarking, fingerprinting and cover song detection} 
\end{itemize} 
\end{frame} 





\begin{frame}{Audio content analysis - A. Lerch}

  \begin{itemize}
  \item{Fundamentals (signals, sampling, quantization, convolution, blocking, fourier transform, correlation)}
  \item{Instantaneous features (statistical features, spectral features, post-processing, dimensionality reduction)}
    \item{Intensity and Loudness}
    \item{Tonal Analysis (pitch, monophonic pitch detection, polyphonic pitch tracking, tuning estimation, key recognition, chord detection)}
    \item{Temporal Analysis (onset detection, tempo and beat detection, downbeat, rhythm description)}
    \item{Audio Alignment (dynamic time warping, audio/score alignment)}
    \item{Music  Classification (genre, similarity, mood)}
      \item{Audio Fingerprinting}
    \end{itemize}
\end{frame} 


\begin{frame}{Fundamentals of Music Processing - Meinard Muller}
  \begin{itemize}
  \item{Music Representations}
  \item{Fourier Analysis of Signals}
  \item{Music Synchronization}
  \item{Music Structure Analysis}
  \item{Chord Recognition}
  \item{Tempo and Beat Tracking}
  \item{Content-based Audio Retrieval}
  \item{Musically Informed Audio Decomposition}
    \end{itemize}
  \end{frame} 


\begin{frame}{Kadenze online MIR program}

\begin{itemize}
\item{Extracting Information from Music Signals}
\item{Machine Learning for MIR}
\item{Music Information Rertieval Systems} 
\end{itemize} 

\end{frame}


\begin{frame}{Extracting Information from Music Signals}
  \begin{itemize}
  \item{Time, frequency, and sinusoids}
  \item{DFT and Time-Frequency Representations}
  \item{Monophonic Pitch Detection}
  \item{Audio Feature Extraction}
    \item{Rhythmic Analysis} 
    \end{itemize} 
\end{frame}


\begin{frame}{Machine Learning for MIR}

  \begin{itemize}
  \item{Supervised Learning and Naive Bayes Classification}
  \item{Discriminative Classifiers}
  \item{Genre Classification}
  \item{Emotion Recognition and Regression}
  \item{Tags}
  \item{Music Visualization} 
  \end{itemize} 

  \end{frame} 

\begin{frame}{Music Retrieval Systems}

  \begin{itemize}
  \item Query Retrieval 
  \item Polyphonic Alignment and Structure Segmentation 
  \item Chord Detection and Cover Song Identification
  \item Transcription and Sound Source Separation
  \item Audio Fingerprinting and Watermarking
    \end{itemize} 

\end{frame} 


\begin{frame}{Other topics}
  \begin{itemize}
  \item{Optical Music Recognition (OMR)}
  \item{Symbolic Music Retrieval}
  \item{MIR for live music performance}
  \item{Computer-assisted music pedagogy}
  \item{Computational Ethnomusicology}
  \item{MIR for music production}
  \item{Natural language processing for MIR} 
  \end{itemize} 

\end{frame}


\begin{frame}{Notebook break I}

  An important pedagogical advice is to use multiple alternating modes
  of delivery when teaching a concept. Notebooks combine text and
  snippets of code structured in cells that can be executed
  individually. Originally popularized by Mathematica, they are
  nowdays very popular as ways of introducing concepts. For the last
  couple of years I frequently utilize Python notebooks in my teaching
  and throughout this tutorial I will be showing some representative
  examples.
\end{frame}

\begin{frame}{Organization: Discussion/Questions}

  Any thoughts/questions/discussion regarding the first module ``Organization'' ? (5-10 minutes) 
  
\end{frame}

\section{Adapting to a target audience}

\begin{frame}{Adaptation}
Another challenge when teaching MIR is the diversity of the students.
MIR is an interdisciplinary field and that can mean that the students
can come from different disciplines and can have very different backgrounds. 
In this module we will go over some strategies for adapting
the material for different audiences and providing supports
to students with different backgrounds. 
\end{frame}


\begin{frame}{Interdisciplinary Research}
Inherently inter-disciplinary and cross-disciplinary work. Connecting
theme: making computers better understand music to create more
effective interactions with musicians and listeners. 

\begin{columns}
\column{0.5\textwidth}
\begin{itemize} 
\item{Music Information Retrieval} 
\item{Digital Signal Processing} 
\item{Machine Learning} 
\item{Human-Computer Interaction} 
\item{Software Engineering} 
\end{itemize} 

\column{0.5\textwidth}
\begin{itemize}
\item{Artifical Intelligence}
\item{Multimedia}
\item{Robotics} 
\item{Visualization}
\item{Programming Languages} 
\end{itemize} 
\end{columns}
\end{frame}

\begin{frame}{EDI}
\begin{itemize}
\item{{\bf Equity} - everyone is treated fairly and equally}
\item{{\bf Diversity} - value differences}
\item{{\bf Inclusion} - everyone feels supporterd and integrated}
\end{itemize}
As instructors how can we support EDI? 
\end{frame} 




\begin{frame}{Some ideas for supporting EDI}
  \begin{itemize}
\item{Focus and value diversity - for example go over a list of past projects that are as diverse in terms of topic as possible }
\item{Consent form to share student material with future offerings/students}
\item{Highlight examples of researchers from underrepresented groups}
\item{Encourage student participation}
\item{Model constructive and respectful feedback}
\item{Optional educational supports, different style deliverables}
\item{Interdisciplinary field means everyone will be outside their comfort zone}
\item{Peer mentorship}
\end{itemize} 
\end{frame} 


\begin{frame}{Backgrounds}

  MIR students can come from many different backgrounds. Let's look
  at some representative examples and what they might lack:

  \begin{itemize}
  \item{{\bf Computer Science:} good programming and ML skills, not as comfortable with mathematical notation and numerical stuff, no formal music knowledge}
  \item{{\bf Electrical and Computer Engineering:} good mathematical notation
    and numerical programming, maybe ML, not as experienced with general programming, no formal knowledge}
    \item{{\bf Music:} little experience with programming, intimidated by mathematical notation, good knowledge of music, writing skills}
    \end{itemize} 

\end{frame}

\begin{frame}{Music 101}

  Some basic knowledge about music notation and theory is useful in
  understanding a lot of MIR research.

  \begin{itemize}
  \item{Absolute and relative encoding of pitch/intervals}
  \item{Common western music relative rhythm notation}
  \item{Chords/Tonality/Key}
  \item{Very briefly scales, harmony, counterpoint, instrumentation}
   \item{Excellent tutorial ISMIR 2021: Practical Music Theory for MIR researchers}
  \end{itemize}
  \end{frame} 


\begin{frame}{Programming 101-A}

  The biggest challenge with teaching programming to musicians
  is to convince them that it is not that hard. There are many
  excellent tutorials available for any programming language/environment.
  Some thoughts:

  \begin{itemize}
  \item{Ask the students to change the tutorial examples they are encountering to be more about music. For example instead of the finding the maximum of a list of numbers change it to finding the highest note in a melody}
  \item{Music21 is a great environment to introduce Python programming
    to music students as they can easily see and hear the results of symbolic music manipulations}
  \item{Try to provide skeleton code with increasing amounts of student involvement}
    \end{itemize} 
  \end{frame}
  
\begin{frame}{Programming 101-B}

  \begin{itemize} 
  \item{Jupyter notebooks provide means of introducing interactivity and programming}
  \item{Some music students are familiar with visual programming environments like Max/MSP. This can be leveraged to get them going with text-based programming} 
  \end{itemize} 
  
\end{frame}


\begin{frame}{Mathematics 101}

  Understanding of mathematics is a long process and in many cases
  music students or even CS students feel that it is not necessary.
  Some thoughts:

  \begin{itemize}
    \item{Decouple notation from concepts} 
    \item{Three views: concrete toy example, computer code, mathematical expression}
    \item{Read papers with emphasis on writing and notation not content}
    \item{Basic vector/matrix notation from linear algebra, probabilities/stats}
    \item{Khan Academy}
  \end{itemize} 
\end{frame} 


\begin{frame}{Active Learning}

  The term {\bf active learning} is used to describe a variety of
  approaches that are different than the traditional lecture mode
  of delivering courses. Active learning interventions include:

  \begin{itemize}
  \item{Group problem-solving}
  \item{Worksheets completed in class}
  \item{Use of polls/clickers}
  \item{Studio/workshop course design}
  \item{Mock conference/peer review}
  \item{Live coding} 
  \end{itemize} 
\end{frame} 

\begin{frame}{Class Size}

  Class size can have a significant effect on how to deliver a
  course. Smaller class sizes (<30-40 students) allow more
  discussion/interaction and do not requireas much preparation for
  assessment. Bigger class sizes (>40 students) pose different challenges.
  
  Ideas for large class sizes:
  \begin{itemize}
  \item{Labs can function as smaller classes}
  \item{More scaffolding in learning materials}
  \item{More standarized assignments}
  \item{Organized communication in public forums with themes}
  \item{Detailed weekly workplans}
  \end{itemize} 
\end{frame} 


\begin{frame}{Evolving over time}

  It is important to constantly evolve both the content and how it is
  delivered to reflect the rapid changes in MIR research and the surrounding
  software context. 

  \begin{itemize}
  \item{2000 - The birth of Marsyas - self-contained code base for MIR, DSP and ML, C++ from scratch}
    \item{2003 - Weka (and SVMs) comes to the scene for ML} 
    \item{MIREX 2007 - Marsyas submissions (92 citations)}
    \item{2010 - sklearn}
    \item{2014 - librosa,mir\_eval}
    \item{2016 - shift to notebooks for teaching}
    \item{2018 - more and more datasets}
    \item{2019 - Spleeter} 
  \end{itemize} 
\end{frame}


\begin{frame}{Notebook break II}

  Let's look at a notebook for explaining the basics of rhythm notation
  to studens without a formal music background and another one explaining
  matrix factorization for students without previous linear algebra skills. 
  
\end{frame}

\begin{frame}{Adaptation: Discussion/Questions}

  Any thoughts/questions/discussion regarding the second module ``Adaptation'' ? (5-10 minutes) 
  
\end{frame}



\section{Online learning}

\begin{frame}{Online learning}

  Online learning has been around for a long time. Massive Open Online
  Courses rose to prominence around 2012. Despite initial claims of
  signalling the death of brick and mortar university teaching they
  have simply become another mode of learning. Covid and the resulting
  transition to online learning has resulted in a renewed interest in online
  learning. In this module we will discuss online learning strategies
  for effective online teaching of MIR.
\end{frame}

\begin{frame}{MOOC critiques}
  
In fact, the absence of serious pedagogy in MOOCs is rather striking,
their essential feature being short, unsophisticated video chunks,
interleaved with online quizzes, and accompanied by social
networking. - Moshe Vardi

A lot of online learners seek very specific answers to particular
problems they encounter rather than more general learning of concepts.
For example they might look up the syntax of Python list comprehensions
but do not care about the more general ideas of functional programming. 

\end{frame}

%% 50 
\begin{frame}{My personal favorite online learning experience}

  Open Studio is an online learning resource for Jazz that I discoverd
  and have followed regularly throughout the pandemic. It is very different
  than a MOOC. Some observations:

  \begin{itemize}
  \item{As much as I would love to I can't find the time to properly practice}
  \item{Regular live activities - guided practice sessions}
  \item{Sense of community}
  \item{Good production quality but also informal interaction}
  \item{Feeling that everyone is part of a journey}
  \item{Regular free social media presence to attract new members} 
  \end{itemize} 
\end{frame} 

\begin{frame}{Categories of online learners}
   \begin{itemize}
   \item{Auditors - watch all videos, few quizzes/exams}
   \item{Completers - most videos, most quizzes/exams}
   \item{Disengaged - quickly drop the course}
   \item{Sampling - occassionally watch specific lectures}
  \end{itemize} 
\end{frame} 


\begin{frame}{Flipped classroom}

  Traditionally students are introduced to new content in class and
  then work on assignments and projects independently at home. A flipped
  classroom works by introducing the content at home and practice working
  through the material at school. It is a type of blended learning.

  Flipping a classroom requires providing video recording of lectures
  as well as other support material for the students to learn concepts
  at their own pace. Common issue is students not viewing the material.
  In 2018 I taught MIR completely as a flipped course with 2 hours
  of contact per week in which I answered questions, had discussions,
  and did some tutorials. The outcome was mixed: students were much more
  engaged and asked lots of questions but they said they would have
  liked some more traditional lecturing. 
\end{frame} 

\begin{frame}{Recording video}

Video recording is essential for online learning. Unfortunately, the
standards for video quality and production are becomning higher and
higher with all the content creator stuff. Some pragmatic advice:

\begin{itemize}
\item{Video recording full lectures can be useful for review but
  does not work very well for online learning}
\item{Either edit well or not at all}
\item{Shorter (5-7 minutes) videos of concepts are better}
\item{Change of camera - close up, zoom out}
\item{Alternate talking, hand-writing, live-coding, slides}
\item{Scripting and tele-prompting}
\item{Repeated recording} 
\end{itemize} 
\end{frame}

\begin{frame}{Research-enriched teaching}

  Research-enriched teaching is a term used to describe
  university level teaching in which the experience of conducting research is
  interleaved with the teaching of concepts.

  \begin{itemize}
  \item{Present papers by previous students in the course}
  \item{Invite as guest lectures previous students}
  \item{Invite industry researchers}
  \item{Present industry-led papers}
  \item{Connect to commercial products}
  \item{Use paper templates for project reports}
  \item{Conference style peer reviewing} 
  \end{itemize} 
  
\end{frame} 

\begin{frame}{Scaffolding}

  Scaffolding provides multiple supports in the form of worked
  examples, incremental increase of difficulty, live worked examples,
  and experimentation to assist with learning of concepts. It also
  refers to breaking a complicated concept into smaller units.
  At the extreme it becomes hand holding or spoon feeding. 

  \begin{itemize}
  \item{Show and tell - active coding}
  \item{Tap into prior knowledge}
  \item{Give time to talk}
  \item{Pre-teach vocabulary/notation}
  \item{Pause, pause, review, pause} 
  \end{itemize} 
  
  \end{frame} 

\begin{frame}{Expanding audience - Initial experiment}

  In the Spring of 2014 I started video recording my MIR
  lectures (simply doing screencasting and Google Hangouts/YouTube)
  and invited external participants to the course. Overall it was
  a positive experience despite the low production quality with
  more than 300 external participants engaging with the course
  by viewing videos and participating in discussion. No assessment/work
  was offered to the external participants. 
  
\end{frame}

\begin{frame}{Expanding audience - Kadenze course}

  This experience led to the development of a full online
  MIR program consisting of 3 courses offered through Kadenze Inc.
  \url{https://www.kadenze.com/programs/music-information-retrieval}
  The videos were recorded in the Spring of 2016 but the course was
  finally made public in 2020.

  {\bf Challenges:} contract agreement with university, privacy issues,
  grading issues 
  
\end{frame} 

\begin{frame}{Notebook break III}

  An alternative to video recordings is providing a more complete self-contained
  detailed notebook. The FMP (Foundations of Music Processing) notebooks are a
  great example of such a teaching resource. Let's look at the notebook
  for music syncronization as an example. 
  
\end{frame}

\begin{frame}{Online Learning: Discussion/Questions}

  Any thoughts/questions/discussion regarding the second module ``Online Learning'' ? (5-10 minutes) 
  
\end{frame}


\section{Assessment}

% degree of difficulty labeling, self-reflection, peer review
% from model solutions to peer critiques, project scaffolding,
%

\begin{frame}{Assessment}

  Assessment is a big and unavoidable challenge in teaching.
  Ideally, assessment should be comprehensive and properly capture
  the knowledge and ability of students in a particular topic. The more
  multi-faceted this assessment is, the more likely it is that it will
  correlate better with real world performance. It is important to talk
  to students about assessment and share how difficult it is. It is also
  important to differentiate between assessment and feedback although
  obviously they are related. 

  
\end{frame}


\begin{frame}{Assessment Overfitting}

  Standarized assessment especially face-to-face timed exams can be
  poor indicators of actual knowledge or performance. At the same time
  they are less prone to academic integrity violations and perceived
  as more objective and fair. The current trend in education research
  is to emphasize learning outcome and well defined rubrics there is a
  dange of putting too much effort into teaching for the assessment
  not for the knowledge.  Motivation, curiosity, and creativity are
  not fosted when only material that is going to be in the exam is
  covered. I call this assessment overfitting in analogy to ML
  overfitting. It is more likely to happen with a flipped classroom
  style of instruction.

\end{frame}

\begin{frame}{Academic Integrity}

  Academic integrity violations are increasing with the shift to online
  teaching.

  \begin{itemize}
  \item{Contract cheating}
  \item{Public code repositories (github)}
  \item{Honest and pragmatic discussion impossible}
  \item{Build trust and respect rather than blame}
  \item{Copy but understand - self respect is critical}
  \end{itemize} 

\end{frame} 

\begin{frame}{Degree of Difficulty (my magic bullet)}

An important and fundamental challenge with assessment is calibration
to student ability. If the tasks are too hard or too easy that
leads to frustration. It is important to stress that each student
is unique and has their own story. Degree of difficulty labeling
refers to explicity labeling assessed work (exam questions, assignments,
quizzes) and even reading materials in terms of difficulty. 

\begin{itemize}
\item{Basic ($40\%$), Expected ($50\%$), Advanced ($10\%$)}
\end{itemize}
  
\end{frame}



\begin{frame}{Self-reflection}

  A good way to ensure personalized course work is to include
  some self-reflection activities. These could be in the form
  of a blog, journal, discussion board. Reading papers and code
  and writing about what was read is another good strategy.
  One of the challenges with self-reflection is that it takes more
  time to grade and does not scale easily to large classes. 
  
\end{frame}

\begin{frame}{Peer review}

  Peer review is a great way to make the students take more ownership of
  their work and learn to provide constructive feedback. It can be challenging
  to administer in large courses. Also some institutions have restrictions on
  using student feedback for grading. 

  
\end{frame}
  
\begin{frame}{From Model solution to peer exposure and review}

  The classic approach for assignments is to provide a model solution.
  A better approach is to critique (anonymous) student work and contrast
  different possible solutions. It also useful for students to contrast
  their work with other students especially when the assessment is subjective.
\end{frame}

\begin{frame}{Auto-grading}

  Auto-grading is trivial with simple questions such as multiple choice
  or keyword matching quizzes. However, it becomes quite more complex
  and challenging as the deliverables get more complicated. A more
  complex setup is to have various unit tests for a piece of code
  (some shared with students and some withheld). My ideal situation
  would be to have a problem generator that personalizes assignments
  to students followed by more flexible similarity checking. For example
  a monophonic melody could be randomly generated using samples, then
  automatic pitch extraction could be applied, followed by comparing the
  results with a state-of-the art pitch detectors. The similarity will
  have to be calibrated but MIR techniques can be used for this purpose. 
  
\end{frame}

\begin{frame}{Rubrics and feedback}

A well defined rubric defines how a deliverable will be graded in a
detailed way. It can help students, instructors, and teaching assistants
be fair and consistent. It also can be used to provide feedback to the student
if it is well designed. It is extremely challenging to design good rubrics.  
  
\end{frame}

\begin{frame}{Gameification}

  Awards, prizes, badges - recycling of journals, student voting 
  
\end{frame}




\begin{frame}{Notebook break IV}

  For assignment a effective possibility is to base them on existing
  notebook/code resources. Let's look at a particular example for
  monophonic pitch detection from the Kadenze program. 
  
\end{frame}

\begin{frame}{Assessment: Discussion/Questions}

  Any thoughts/questions/discussion regarding the second module ``Assessment'' ? (5-10 minutes) 
  
\end{frame}




\section{Projects}

\begin{frame}{Group projects}
  The most interesting for the students and most challenging aspect
  of teaching a MIR course is the completion of a group project. One of my
  most satisfying experiences as a MIR teacher has been to see some of these
  projects evolve into successful ISMIR papers. In this module, we will
  go over some of the logistics of group projects, overview some examples,
  and discuss instructor strategies for making them successful. 
  
\end{frame}


\begin{frame}{Group formation and size}

  Group projects are challenging but rewarding. Optimal group size can vary.
  In small classes groups of 2 are feasible. Groups of 3/4 students are more
  common with larger classes. Bigger groups might be a necessity with really
  large classes but almost always create work balance tensions. Some ideas:

  \begin{itemize}
  \item{Random group formation}
  \item{Student-led group formation}
  \item{Project-based group formation}
  \item{Multi-phase with reshuffling}
  \item{Pre-defined roles/responsibilities}
  \item{Trade-off between detailed specification and creative freedom}
  \end{itemize} 
  
\end{frame}

\begin{frame}{Project deliverables}
  \begin{itemize}
  \item{Design Specification $15\%$}
  \item{Progress Report $15\%$}
  \item{Final report $20\%$}
  \item{Presentation/video $10\%$}

    \end{itemize} 

  \end{frame} 

  
\begin{frame}{Project stages}

  At least one of these stages should be not trivial. With more software
  frameworks and datasets it is easier to create more sophisticated projects. 

  \begin{itemize}
  \item{Problem specification, data collection and ground truth annotation}
  \item{Information extraction and analysis}
  \item{Implementation and interaction}
  \item{Evaluation} 
  \end{itemize} 
  
\end{frame}


\begin{frame}{Example projects}

  Several projects in the UVic MIR course evolved into ISMIR publications.
  Some examples:
  
  \begin{itemize}
  \item{Query-by-beatboxing - 2006}
  \item{Stereo Panning - 2008}
  \item{Examining DJ ordering of playlists - 2013}
  \item{Curriculum learning for automatic chord recognition (2021)}
    
  \end{itemize} 

  
\end{frame} 

\begin{frame}{Notebook break V}
  Matrix factorization notebook 
\end{frame}

\begin{frame}{Projects: Discussion/Questions}

  Any thoughts/questions/discussion regarding the second module ``Projects'' ? (5-10 minutes) 
  
\end{frame}



\section{Resources}

\begin{frame}{Teaching Resources}

  As MIR evolves as a research field, there are more and more
  resources such as books, overview articles, software frameworks and
  tools, and datasets that can be used to support teaching. In this module,
  we go over some of my personal favorites. This is by no means a comprehensive
  list of resources but rather some representative examples from what I use
  in my own teaching with some discussion of what I like about them. 

  \end{frame} 


\begin{frame}{Books}
  \begin{itemize}
    \item Fundamentals of Music Processing - Meinard Muller 
    \item Audio content analysis - Alexander Lerch
    \item Music Data Mining - Li, Ogihara, Tzanetakis (editors) 
    \item Digital Signal Processing Primer - Ken Steiglitz
    \item Music Similarity and Retrieval - Markus Schedl
    \end{itemize}
\end{frame}

\begin{frame}{Overview papers}

  Good overview papers can be a great resources for MIR students.
  A favorite example:
  ``Automatic chord estimation from audio: A review of the state of the art'', 2014. M McVicar, R Santos-Rodríguez, Y Ni, T De Bie

  

\end{frame} 

\begin{frame}{MIR Software Frameworks}
  \begin{itemize}
  \item{Essentia}
  \item{MIR Toolbox}
  \item{librosa}
  \item{Marsyas}
  \item{Music21} 
  \end{itemize}
  
\end{frame}

\begin{frame}{MIR Task Software}
  \begin{itemize}
  \item{Spleeter}
  \item{musicnn}
  \item{madmom}
  \item{melodia}
  \end{itemize} 
\end{frame}


\begin{frame}{MIR Tools}
  \begin{itemize}
  \item{mirdata}
  \item{jams}
  \item{mireval} 
  \end{itemize} 
\end{frame} 


\begin{frame}{MIR GUIs}
  \begin{itemize}
  \item{Sonic Visualizer/Vamp plugins}
  \item{Audacity}
  \item{Tony}
  \item{DAWs such as Ableton Live, Reaper, Plugins}
  \item{Max/MSP, PureData}
  \item{Game Engines}
  \end{itemize} 
\end{frame}

\begin{frame}{Other Software}
  \begin{itemize}
    \item{Weka} 
    \item{scikit-learn}
    \item{tensorflow}
    \item{pytorch}
    \item{pandas,numpy,scipy}
    \item{Juce} 
    \end{itemize} 
\end{frame} 



\begin{frame}{Datasets}
  \begin{itemize}
  \item{GTZAN (historic)}
  \item{FMA (large, medium, small)}
  \item{Magnatagatune}
  \item{Music4All}
  \item{MTG-JAMENDO}
  \item{mir\_data}
  \end{itemize}
\end{frame}



\begin{frame}{Reproducible research}
  \begin{itemize}
  \item{Provide code, data, plots}
  \item{Work on multiple machines and with multiple people}
  \item{It takes a lot of effort and time but can be worth it}
  \item{Do not and do re-invent the wheel}
  \item{Be part of an ecosystem}
  \item{Open source license}
  \item{Public source repository}
  \item{Documentation, tutorials, examples}
  \item{Marsyas: 581 citations, librosa: 1251 citations, scikitlearn: 47321, essentiaL: 431}
  \end{itemize} 
\end{frame} 

\begin{frame}{Teaching Materials}
  \begin{itemize}
  \item{\url{https://musicinformationretrieval.com/}}
  \item{\url{https://ismir.net/resources/tutorials/}}
  \item{\url{https://www.audiolabs-erlangen.de/resources/MIR/FMP/C0/C0.html}}
  \item{\url{https://www.audiocontentanalysis.org/}}
  \end{itemize} 
\end{frame}


\begin{frame}{Environments}

  \begin{itemize}
  \item{Jupyter notebooks}
  \item{Hosted environments: JupyterHub, Google Colab}
  \item{Github}
  \item{Overleaf}
  \end{itemize} 

  \end{frame} 

\begin{frame}{Notebook break VI}

  THX Logo Notebook 
\end{frame}

\begin{frame}{Resources: Discussion/Questions}

  Any thoughts/questions/discussion regarding the second module ``Resources'' ? (5-10 minutes) 
  
\end{frame}



\section{Conclusions}

\begin{frame}{Conclusions}

  \begin{itemize}
  \item Teaching is a journey not just a destination
  \item Teaching is garden not just a journey
  \item Experiment and question assumptions
  \item Be honest and pragmatic with students
  \item Focus on intuition and motivation
  \item Teaching is more about pacing and filtering and less about content delivery
  \end{itemize} 
\end{frame}
  


\setbeamercolor{bibliography item}{fg=black}
\setbeamercolor*{bibliography entry title}{parent=palette primary}

\bibliographystyle{amsalpha}
\bibliography{dhsi_2014} 




\end{document}











